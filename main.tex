%%
%% Copyright 2007-2020 Elsevier Ltd
%%
%% This file is part of the 'Elsarticle Bundle'.
%% ---------------------------------------------
%%
%% It may be distributed under the conditions of the LaTeX Project Public
%% License, either version 1.2 of this license or (at your option) any
%% later version.  The latest version of this license is in
%%    http://www.latex-project.org/lppl.txt
%% and version 1.2 or later is part of all distributions of LaTeX
%% version 1999/12/01 or later.
%%
%% The list of all files belonging to the 'Elsarticle Bundle' is
%% given in the file `manifest.txt'.
%%
%% Template article for Elsevier's document class `elsarticle'
%% with harvard style bibliographic references

%\documentclass[preprint,12pt,authoryear]{elsarticle}

%% Use the option review to obtain double line spacing
%% \documentclass[authoryear,preprint,review,12pt]{elsarticle}

%% Use the options 1p,twocolumn; 3p; 3p,twocolumn; 5p; or 5p,twocolumn
%% for a journal layout:
%% \documentclass[final,1p,times,authoryear]{elsarticle}
%% \documentclass[final,1p,times,twocolumn,authoryear]{elsarticle}
%% \documentclass[final,3p,times,authoryear]{elsarticle}
%% \documentclass[final,3p,times,twocolumn,authoryear]{elsarticle}
%% \documentclass[final,5p,times,authoryear]{elsarticle}
 \documentclass[final,5p,times,twocolumn,authoryear]{elsarticle}

%% For including figures, graphicx.sty has been loaded in
%% elsarticle.cls. If you prefer to use the old commands
%% please give \usepackage{epsfig}

%% The amssymb package provides various useful mathematical symbols
\usepackage{amssymb}
%% The amsthm package provides extended theorem environments
%% \usepackage{amsthm}

\usepackage{url}

%% The lineno packages adds line numbers. Start line numbering with
%% \begin{linenumbers}, end it with \end{linenumbers}. Or switch it on
%% for the whole article with \linenumbers.
%% \usepackage{lineno}


\begin{document}

\begin{frontmatter}

\input{sections/AA_cover} % Title page setup

\begin{abstract}
This article presents a machine learning-based approach to profile student performance in the PISA 2022 dataset, with a specific focus on students enrolled in vocational and professional education tracks. The study aims to identify the most influential predictors of low mathematics achievement and explore the role of psychological and contextual factors, such as socioeconomic status and school-related anxiety, in shaping academic outcomes. Our pipeline involves data preprocessing (OneHotEncoding and standard scaling), unsupervised clustering with KMeans to segment students into performance tiers, and a supervised learning phase using Random Forest classifiers to compute feature importances. These insights contribute to the development of targeted, evidence-based interventions to reduce underperformance in diverse educational systems.
\end{abstract}

\begin{keyword}
PISA \sep Vocational education \sep Educational data mining \sep Random Forest \sep KMeans \sep Student profiling \sep Academic performance \sep School anxiety
\end{keyword}

\end{frontmatter}

%\tableofcontents

%% \linenumbers

%% main text

\section{Introduction}
\label{introduction}

\subsection{Business Understanding}
\label{sec:business_understanding}

Large-scale international assessments like the \textbf{Programme for International Student Assessment (PISA)} provide essential data to explore the structural and psychological factors that influence student achievement. Our project leverages the \textbf{PISA 2022 Student Questionnaire} dataset to develop a model capable of identifying students at high risk of academic underperformance. However, in this article, I focus on a particular group of students (those enrolled in \textbf{vocational and professional training tracks}).

\paragraph{Policy problem}
Vocational-track students are a critical, yet often overlooked, subgroup within secondary education. They account for a significant proportion of 15-year-olds internationally, and their mathematics scores tend to cluster in the lower performance deciles. This study seeks to address this issue through the following central question:

\begin{quote}
  \emph{“How can we identify the most relevant factors contributing to low mathematics performance among vocational students, and how can these insights inform more equitable and effective policy design?”}
\end{quote}

\paragraph{Focus and objectives}
Guided by this question, this project sets out to:

\begin{enumerate}
  \item \textbf{Profile vocational students}: Identify distinguishing academic, social, and attitudinal characteristics of students enrolled in vocational tracks.
  \item \textbf{Cluster performance patterns}: Use unsupervised learning to differentiate between low-, medium-, and high-achieving vocational students.
  \item \textbf{Rank risk indicators}: Leverage supervised learning to compute and rank variable importance for early detection and intervention.
\end{enumerate}

\paragraph{Vocational-track scope}
Based on early analysis of the PISA 2022 dataset, I found:

\begin{itemize}
  \item The student population totals 613,744 individuals.
  \item Of these, \textbf{75,660 students (approx. 12.3\%)} are enrolled in vocational programs (classified under ISCEDP codes 254, 354, 453).
  \item The remaining 538,080 students are in general education tracks.
  \item Vocational students are predominantly concentrated in Grades 9–11, with Grade 10 comprising the largest share.
\end{itemize}

These patterns underscore the importance of segment-specific analysis. A one-size-fits-all model is insufficient for understanding the unique needs of vocational learners.

\paragraph{Psychological factors}
Anxiety remains a major obstacle to performance, especially for highly skilled students. According to prior findings from \cite{anxietyPisa}, school- and test-related anxiety disproportionately hinders top-performing students across all subjects. Even when controlling for socioeconomic and demographic variables, anxiety alone contributed to statistically significant score reductions. This phenomenon should also be considered in vocational-track students, who frequently report lower academic self-efficacy and fewer support mechanisms.

\paragraph{Cross-country considerations}
Although PISA enables cross-national comparisons, educational systems vary widely:

\begin{description}
  \item[Germany:] Maintains one of the most structured vocational education systems, with well-established dual-track integration.
  \item[Singapore:] Offers technical tracks with competitive admissions and strong government oversight.
  \item[England:] Provides post-16 vocational pathways but tracks relatively few students before GCSEs.
  \item[China:] Vocational education is common but often perceived as a fallback, with lower societal status.
  \item[United States:] CTE programs vary by state and are currently undergoing reform to boost parity with academic pathways.
\end{description}

These differences necessitate the use of fixed effects and interaction terms at the system level in the modeling strategy.

\paragraph{Success criteria}
This phase of the CRISP-DM framework is considered complete when:

\begin{itemize}
  \item A profile of at-risk vocational students is clearly defined and empirically grounded;
  \item Predictive models reliably differentiate between low, average, and high performers;
  \item Stakeholders can use feature importances to inform targeted policy actions;
  \item Psychological barriers like anxiety are considered in risk assessment.
\end{itemize}

\paragraph{Model foundation}
To meet these goals, I implemented a multistage pipeline:

\begin{itemize}
  \item In the \textbf{pre-processing phase}, I applied \texttt{OneHotEncoding} to categorical variables and \texttt{StandardScaler} to numerical features, ensuring compatibility across clustering and classification tasks.
  \item For \textbf{unsupervised learning}, I used \texttt{KMeans} clustering with three target clusters: below-average, average, and above-average math performers. This revealed natural groupings in the data that aligned with performance strata.
  \item In the \textbf{supervised learning phase}, I trained a \texttt{Random Forest Classifier} to classify math performance categories and derive feature importances. These importances will help prioritize variables for educational intervention and policy action.
\end{itemize}

This integrated approach—blending interpretability with predictive power—positions our model to serve as a foundation for scalable, equity-focused education analytics.

\section{Data Understanding}
\label{sec:data_understanding}

\subsection{Data Collection}
This study relies on the \textbf{PISA 2022 Student Questionnaire dataset}, which was downloaded from the OECD public data portal\footnote{\url{https://webfs.oecd.org/pisa2022/index.html}}. The original file was provided in SAS format (\texttt{.sas7bdat}) and subsequently converted into CSV for processing in Python using \texttt{pandas}.

\subsection{Data Overview}
The complete dataset contains responses from \textbf{613,744 students} across PISA-participating countries and includes \textbf{1278 variables}. These span a wide array of features, including demographic characteristics, home environment, school experience, psychological indicators, and academic results.

After loading and inspecting the data, I found the following:

\begin{itemize}
  \item The dataset includes \textbf{1274 numeric variables} and only \textbf{4 categorical variables}.
  \item Variables cover socio-economic status (ESCS), parental education, learning time, school infrastructure, self-beliefs, and emotional well-being.
\end{itemize}

\subsection{Vocational Student Segmentation}
To address this study’s specific focus, I filtered for students in vocational and professional tracks using the \texttt{ISCEDP} classification codes (254, 354, 453). This segmentation revealed the following.

\begin{itemize}
  \item \textbf{75,660 students} (approximately \textbf{12.3\%}) are enrolled in vocational programs.
  \item The remaining \textbf{538,080 students} are in general education tracks.
\end{itemize}

When breaking down vocational students by grade level (variable \texttt{ST001D01T}), we observed that the majority are in:

\begin{itemize}
  \item Grade 10 (most frequent),
  \item Followed by Grade 9 and Grade 11.
\end{itemize}

This distribution aligns with age-based expectations, as PISA tests 15-year-olds regardless of national curriculum grade level.

\subsection{Variable Structure}
An initial classification of feature types revealed the dominance of continuous numerical variables. Given the high dimensionality of the dataset, subsequent preprocessing steps (discussed in Section \ref{sec:data_preparation}) involve standardization and encoding procedures to enable effective machine learning analysis.

Particular attention was paid to the following dimensions:
\begin{itemize}
  \item \textbf{Academic outcomes}: PISA math scores served as the basis for clustering and classification.
  \item \textbf{Psychosocial variables}: Indicators such as anxiety, confidence in problem solving, and sense of belonging were included to assess latent psychological burdens on performance.
  \item \textbf{Demographic segmentation}: Gender, parental education, and household resources were included to contextualize disadvantage.
\end{itemize}

\subsection{Limitations and Considerations}
Although the dataset is rich and globally representative, several challenges emerged:

\begin{itemize}
  \item The high number of features poses dimensionality and interpretability issues.
  \item Some values were missing or inconsistently formatted, which required cleaning or imputation during preprocessing.
  \item Country-specific educational systems introduce variability in grade levels and program structures, which will be taken into account in later modeling phases via fixed effects and interaction terms.
\end{itemize}

In general, the data provides a comprehensive and well-structured foundation for segmenting students, especially those in vocational programs, by risk of low performance and identifying targeted variables that inform the intervention.


\section{Data Preparation}
\label{sec:data_preparation}

\subsection{Data Cleaning}
The initial dataset included over 1,200 variables, many of which were sparsely populated. To ensure robustness, I excluded all variables with more than \textbf{70\% missing values}. Additionally, students from \texttt{England (GBR)} were removed due to known structural differences in their education system that could confound cross-country comparisons.

\subsection{Feature Construction}
Each subject area in the PISA dataset (math, reading, and science) provides 10 \textit{plausible values} (PVs) to account for measurement uncertainty. We computed the mean of these PVs to generate a single, interpretable performance metric per subject, simplifying downstream modeling. Specifically:

\begin{itemize}
  \item \textbf{Avg Math Result} was used as the principal academic outcome for clustering and classification tasks.
  \item \textbf{Avg Reading Result} and \textbf{Avg Science Result} were computed similarly for auxiliary analysis.
\end{itemize}

We also removed finer-grained math subscale variables, as their influence is already embedded in the aggregate math scores.

\subsection{Data Formatting}
To address the high cardinality of the \texttt{CNT} (Country) variable, we grouped countries into three performance tiers—\textbf{Above Average}, \textbf{Average}, and \textbf{Below Average}—based on their national mean in math performance, using a ±15 point tolerance from the global average.

This transformation preserved important educational context while reducing overfitting risk in machine learning models due to country-specific noise.

\subsection{Integration}
No further data integration was required. The dataset already contains all relevant variables in a unified structure.

\vspace{1em}
The resulting dataset is significantly cleaner, more compact, and analytically tractable—ready for exploratory modeling and predictive analysis in subsequent phases of this study.

\section{Modeling}
\label{sec:modeling}

\subsection{Preprocessing}
To prepare the data for clustering and classification, both categorical and numerical features were processed via a dedicated pipeline. Numerical variables were imputed using mean values and standardized using \texttt{StandardScaler}, while categorical variables underwent mode imputation followed by one-hot encoding. This ensured all features contributed equally to distance-based algorithms and prevented bias due to differing scales or missing entries.

The transformed dataset retained a wide range of psychosocial, academic, and demographic indicators. After preprocessing, we focused our analysis on the subset of students enrolled in vocational programs, based on \texttt{ISCEDP} codes (254, 354, 453), as defined earlier.

\subsection{Clustering Approach}
We applied \textbf{K-Means clustering} to both vocational and general education subsets to explore latent groupings among students. The optimal number of clusters was empirically set to \textbf{3} for each group, reflecting a reasonable balance between model complexity and interpretability.

\subsection{Cluster Analysis}
Clustering revealed meaningful differentiation in performance profiles:

\begin{itemize}
  \item Among \textbf{vocational students}, three clusters emerged with distinct average math performance levels:
    \begin{itemize}
      \item \textbf{Cluster 0}: Lower performers (mean $\approx$ 353, std $\approx$ 35),
      \item \textbf{Cluster 1}: Mid-range performers (mean $\approx$ 466, std $\approx$ 33),
      \item \textbf{Cluster 2}: Higher performers (mean $\approx$ 562, std $\approx$ 36).
    \end{itemize}

  \item For \textbf{non-vocational students}, a similar triad was observed:
    \begin{itemize}
      \item \textbf{Cluster 0}: Low (mean $\approx$ 371),
      \item \textbf{Cluster 1}: Mid (mean $\approx$ 474),
      \item \textbf{Cluster 2}: High (mean $\approx$ 565).
    \end{itemize}
\end{itemize}

This symmetry suggests that despite curricular differences, both student groups exhibit a common structure in performance distribution, possibly driven by shared socio-educational determinants.

\subsection{Dimensionality Reduction to Interpret Clustering Results}
To visually interpret clustering results, we employed \textbf{Principal Component Analysis (PCA)} to project the high-dimensional feature space into two principal components. The resulting 2D scatter plots, colored by cluster label, show distinct and well-separated clusters—particularly within the vocational subset—confirming that the clustering captured meaningful structure in the data.

\subsection{Clustering Interpretation and Implications}
The stratification of students into performance-based clusters provides a foundation for targeted interventions. Notably, a non-trivial proportion of vocational students fall into the low-performing cluster. Further work will involve analyzing the predictors that most strongly differentiate these clusters, including self-beliefs, learning time, and emotional well-being, with the aim of identifying modifiable risk factors.

This clustering demonstrates the feasibility of segmenting at-risk students in vocational tracks using unsupervised learning techniques, paving the way for predictive modeling and policy recommendation in later stages.

\subsection{Supervised Modeling: Random Forest Classifier}
To complement the unsupervised clustering, \textbf{Random Forest classifiers} were trained to predict cluster membership for both vocational and non-vocational students, using the same set of preprocessed features. The aim was to understand which variables most strongly distinguish between the low, medium, and high academic performance clusters identified by the K-Means algorithm. The models were trained with default hyperparameters.

\subsection{Feature Importance}
The Random Forest models provide insight into the relative importance of variables in classifying students into the different performance clusters. The feature importances were calculated for each cluster (0, 1, and 2) for both the vocational and non-vocational student groups, as well as for predicting the overall cluster assignment. The top predictive features consistently observed across different analyses included:

\begin{itemize}
    \item \textbf{Self-efficacy in math}: Consistently appeared as a strong predictor, particularly in distinguishing higher-performing clusters.
    \item \textbf{Disciplinary climate at school}: Showed significant importance, suggesting the school environment plays a role in cluster membership.
    \item \textbf{Teacher support}: Indicated as a relevant factor in differentiating student groups.
    \item \textbf{Parental involvement}: Features related to parental support and education level were important predictors.
    \item \textbf{Learning time at home}: While mentioned in the original text, its importance varied across clusters and groups.
\end{itemize}

These findings reinforce the multifactorial nature of student outcomes, highlighting the combined influence of individual factors, school environment, and home support on academic performance clusters.

\subsection{Random Forest Classifier Interpretation}
The supervised models confirm that cluster assignment can be reliably predicted from the questionnaire data for both vocational and non-vocational students. This suggests a clear path for identifying students likely to fall into specific performance groups based on readily available information. Educational policymakers could potentially use such models for early identification and targeted intervention strategies. For example, focusing on improving self-efficacy in math or fostering a positive disciplinary climate could potentially benefit students at risk of being in lower-performing clusters.

% \section{Methods}
% %%\label{}

% \subsection{Stacking}
% Stacking is an ensemble learning technique developed by \cite{WOLPERT1992241} that combines several prediction models and uses their outputs as input for a next-level model (meta-model), aiming to improve overall prediction performance\cite{Stacking}. In the study \cite{Stacking}, the authors divide stacking into level 0 and level 1, where layer 0 refers to where the different models generate distinct predictions, and level 1 combines predictions of level 0. For level 0, some of the different models that could be used are DTs, NNs, SVMs, and kNN, in which the resulting predictions are then used and combined in layer 1, in order to improve the overall prediction \cite{WOLPERT1992241}, the authors choose to use ridge regression, which can be useful to avoid overfitting \cite{CUI2021107038}
% \subsection{Boosting}
% Because the dataset used in this study is composed of missing data, it is imperative to implement learners that deal with missing data, the boosting algorithms used were XGBoost, HGB, and LightGBM \cite{Stacking}
% \subsection{blending}
% Blending and stacking are both techniques used in ensemble learning that share similarities. The key difference between them is that in blending, the base learners are trained using predictions obtained from the validation set instead of directly from the training set\cite{Stacking}


\section{Results}
\label{sec:results}
This section presents the findings from the analysis aimed at profiling students in vocational and professional courses in PISA, identifying variables that explain their performance in mathematics, and examining trends across assessment cycles. While the initial analysis focused on a specific dataset and its limitations, the methodology applied provides insights into potential contributing factors to mathematics performance.

\subsection{Student Profiling using K-Means Clustering}
To understand the diverse profiles of students, K-Means clustering was applied to the preprocessed data. Recognizing the feedback to analyze students without prior division, clustering was initially performed on the entire dataset. This allowed for the identification of distinct groups of students based on their characteristics across various features. Subsequently, the distribution of vocational and non-vocational students within these overall clusters was examined to understand where these specific student profiles tend to be located in terms of the identified segments.

Following this initial exploration, separate K-Means clustering was performed on the vocational and non-vocational student groups. For the vocational group, three clusters were identified. An examination of the mean "Avg Math Result" within these clusters revealed significant differences:

\begin{itemize}
    \item \textbf{Cluster 0:} Exhibited the lowest average mathematics result, indicating a lower-performing group.
    \item \textbf{Cluster 1:} Showed a moderate average mathematics result.
    \item \textbf{Cluster 2:} Achieved the highest average mathematics result, representing a higher-performing group.
\end{itemize}

\begin{figure}
	\centering
	\includegraphics[width=0.4\textwidth]{figures/kmeans_clusters.png}
	\caption{Shows the number of clusters revealed by K-Means separated by vocational and non-vocational students.}
    \label{fig:kmeans_clusters}
\end{figure}

Similarly, three clusters were identified for the non-vocational group, and a comparable pattern of low, medium, and high average mathematics results was observed across these clusters (found in Figure \ref{fig:kmeans_clusters}). The box plots visualizing the distribution of "Avg Math Result" for each cluster within both vocational and non-vocational groups (found in Figure \ref{fig:kmeans_clusters_avg_math}) further illustrate these performance differences and confirm the cluster separation based on mathematics achievement. To avoid overwhelming the reader, detailed visualizations for each feature across clusters for vocational students, as generated in the notebook, will be placed in the appendix.

\begin{figure}
	\centering
	\includegraphics[width=0.4\textwidth]{figures/kmeans_clusters_avg_math.png}
	\caption{Shows the Average Math Result for each cluster revealed by K-Means separated by vocational and non-vocational students.}
    \label{fig:kmeans_clusters_avg_math}
\end{figure}

The clustering results begin to answer the question about the profile of students by demonstrating that within both vocational and non-vocational tracks, there are distinct subgroups with varying levels of mathematics performance.

\subsection{Identifying Variables Explaining Mathematics Performance}
To understand which variables contribute to explaining mathematics performance within the identified clusters, supervised learning using a Random Forest classifier was employed. The goal was to determine which features were most important in distinguishing students belonging to the different performance clusters (you will find the figures for each individual cluster in the \ref{appendix:figures}).

Feature importance analysis was conducted for predicting membership in each cluster (0, 1, and 2) for both vocational and non-vocational students, as well as for predicting overall cluster assignment. The results consistently highlighted several key variables as strong predictors of cluster membership (Figure \ref{fig:feature_importances_overall} shows the top feature importances overall). These variables provide insights into the factors associated with different levels of mathematics performance:

\begin{itemize}
    \item \textbf{Self-efficacy in math:} This consistently appeared as a highly important feature in distinguishing clusters, particularly in separating higher-performing groups from lower-performing ones. This suggests that a student's belief in their ability to succeed in mathematics is a significant contributor to their actual performance.
    \item \textbf{Disciplinary climate at school:} The perceived disciplinary environment within the school also emerged as a significant predictor. A positive and orderly school climate appears to be associated with better mathematics performance and membership in higher-performing clusters.
    \item \textbf{Teacher support:} The level of support students feel they receive from their teachers was another important factor. This indicates that supportive teacher-student relationships can play a role in a student's mathematics achievement.
    \item \textbf{Parental involvement:} Features related to parental support and education level were found to be important in predicting cluster membership. This underscores the influence of the home environment on academic outcomes.
\end{itemize}

\begin{figure}
	\centering
	\includegraphics[width=0.4\textwidth]{figures/feature_importances_overall.png}
	\caption{Shows the combined Top Feature Importances for all clusters revealed by Random Forest classifier (vocational vs. non-vocational students)}
    \label{fig:feature_importances_overall}
\end{figure}

These findings directly address the question about variables that contribute to explaining mathematics performance by identifying specific factors that differentiate students across performance levels within both vocational and non-vocational pathways. The feature importance analysis indicates that a combination of individual characteristics (self-efficacy), school-related factors (disciplinary climate, teacher support), and home environment factors (parental involvement) are key contributors to mathematics performance.

\subsection{Evolution over Assessment Cycles}
Due to the limitations of the dataset used in this analysis, which appears to be from a single assessment cycle, it is not possible to directly assess how the profile of students in vocational and professional courses and the variables explaining their performance in mathematics have evolved over various assessment cycles based on this specific analysis. To address this part of the research question, a longitudinal analysis incorporating data from multiple PISA assessment cycles would be required. This would allow for the examination of trends in student demographics, performance levels, and the impact of various factors over time.

\subsection{Connecting Results to Research Questions and Future Directions}
The K-Means clustering and subsequent feature importance analysis provide a valuable initial profile of students in vocational and professional courses by segmenting them into distinct performance groups and identifying key characteristics associated with each group. The analysis highlights that factors beyond just the chosen educational track influence mathematics performance, with individual, school, and home environment variables playing significant roles.

While this analysis provides insights into the "What variables contribute to explaining their performance in mathematics?" question within the scope of the available data, the absence of data from multiple assessment cycles prevents a direct answer to the "How has it evolved over the various assessment cycles?" question. Future research should focus on incorporating longitudinal PISA data to track these changes over time.

Furthermore, enriching the student profiles with demographic information at a country level, as the instructors suggested, could provide valuable context for comparative analysis between different countries and their respective vocational and non-vocational education systems. This would allow for a deeper understanding of how national contexts and policies might influence student profiles and mathematics performance.

\section{Conclusion}

This analysis aimed to profile students in vocational and professional courses in PISA, identify variables contributing to their mathematics performance, and explore the evolution of these factors across assessment cycles. Utilizing K-Means clustering and Random Forest feature importance analysis on a subset of PISA data, we were able to gain insights into the characteristics of students within different performance levels in mathematics.

The clustering results revealed the presence of distinct performance groups within both vocational and non-vocational student populations, demonstrating that students pursuing vocational paths are not a homogeneous group in terms of mathematics achievement. The clear separation of clusters based on average mathematics results highlights the importance of identifying these subgroups for targeted educational support.

The feature importance analysis, derived from the Random Forest models, provided valuable insights into the variables that best distinguish these performance clusters. Consistently across analyses, factors such as self-efficacy in mathematics, the disciplinary climate at school, teacher support, and parental involvement emerged as significant predictors of cluster membership. These findings underscore the multifaceted nature of mathematics performance, emphasizing that a student's success is influenced by a combination of their individual beliefs and skills, the school environment, and the support they receive at home. This directly addresses the question of which variables contribute to explaining mathematics performance within these student profiles.

However, a significant limitation of this analysis, due to the constraints of the available data from a single assessment cycle, is the inability to comprehensively address the question of how these profiles and contributing variables have evolved over time. To fully answer this aspect of the research question, a longitudinal study incorporating data from multiple PISA cycles is essential.

Future research should build upon these findings by:
\begin{enumerate}
    \item Conducting a longitudinal analysis using PISA data from several assessment cycles to track changes in student profiles and the influence of key variables over time.
    \item Incorporating a broader range of demographic and socioeconomic variables, potentially at a country level, to enrich student profiles and facilitate comparative analysis across different educational systems.
    \item Exploring the interactions between different variables to understand how they collectively influence mathematics performance in vocational and non-vocational contexts.
\end{enumerate}

In conclusion, this study provides a foundational understanding of student profiles and factors related to mathematics performance in vocational and professional courses based on a snapshot of PISA data. The identification of key variables contributing to performance differences within these groups offers valuable insights for educators and policymakers seeking to implement targeted interventions. Addressing the limitations regarding the longitudinal analysis and incorporating richer demographic data will be crucial for a more comprehensive understanding of the complex factors influencing mathematics achievement among these students.


%% If you have bibdatabase file and want bibtex to generate the
%% bibitems, please use
%%
\bibliographystyle{elsarticle-harv}
\bibliography{main}

\newpage
\appendix
\section{Additional Figures}
\label{appendix:figures}

This appendix presents supplementary visualizations that support the main analysis in the article.

\begin{figure}[h!]
    \centering
    \includegraphics[width=0.8\linewidth]{figures/bfmj2_box_plot.png}
    \caption{Shows the Father’s occupational status (ISEI) based on 4-digit human coded ISCO, for each cluster, revealed by K-Means, separated by vocational and non-vocational students.}
    \label{fig:father_occupational_status}
\end{figure}

\begin{figure}[h!]
    \centering
    \includegraphics[width=0.8\linewidth]{figures/bmmj1_box_plot.png}
    \caption{Shows the Mother’s occupational status (ISEI) based on 4-digit human coded ISCO, for each cluster, revealed by K-Means, separated by vocational and non-vocational students.}
    \label{fig:mother_occupational_status}
\end{figure}

\begin{figure}[h!]
    \centering
    \includegraphics[width=0.8\linewidth]{figures/cnt_group_box_plot.png}
    \caption{Shows the Country Group, for each cluster, revealed by K-Means, separated by vocational and non-vocational students.}
    \label{fig:country_group}
\end{figure}

\begin{figure}[h!]
    \centering
    \includegraphics[width=0.8\linewidth]{figures/creatoos_box_plot.png}
    \caption{Shows the Creative Activities outside of school (WLE), for each cluster, revealed by K-Means, separated by vocational and non-vocational students.}
    \label{fig:creatoos}
\end{figure}

\begin{figure}[h!]
    \centering
    \includegraphics[width=0.8\linewidth]{figures/escs_box_plot.png}
    \caption{Shows the Index of economic, social and cultural status, for each cluster, revealed by K-Means, separated by vocational and non-vocational students.}
    \label{fig:escs}
\end{figure}

\begin{figure}[h!]
    \centering
    \includegraphics[width=0.8\linewidth]{figures/famcon_box_plot.png}
    \caption{Shows the Subjective familiarity with mathematics concepts (WLE), for each cluster, revealed by K-Means, separated by vocational and non-vocational students.}
    \label{fig:famcon}
\end{figure}

\begin{figure}[h!]
    \centering
    \includegraphics[width=0.8\linewidth]{figures/hisei_box_plot.png}
    \caption{Shows the Highest parental occupational status (ISEI) based on 4-digit human coded ISCO, for each cluster, revealed by K-Means, separated by vocational and non-vocational students.}
    \label{fig:hisei}
\end{figure}

\begin{figure}[h!]
    \centering
    \includegraphics[width=0.8\linewidth]{figures/homepos_box_plot.png}
    \caption{Shows the Home possessions (WLE), for each cluster, revealed by K-Means, separated by vocational and non-vocational students.}
    \label{fig:homepos}
\end{figure}

\begin{figure}[h!]
    \centering
    \includegraphics[width=0.8\linewidth]{figures/ic171q02ja_box_plot.png}
    \caption{Shows How often use out of school: Smartphone (i.e. mobile phone with Internet access), for each cluster, revealed by K-Means, separated by vocational and non-vocational students.}
    \label{fig:smartphone_use}
\end{figure}

\begin{figure}[h!]
    \centering
    \includegraphics[width=0.8\linewidth]{figures/ic173q04ja_box_plot.png}
    \caption{Shows How often use [digital resources] in lessons in [Computer science], [information technology], [informatics] or similar lessons, for each cluster, revealed by K-Means, separated by vocational and non-vocational students.}
    \label{fig:use_of_digital_resources_in_lessons}
\end{figure}

\begin{figure}[h!]
    \centering
    \includegraphics[width=0.8\linewidth]{figures/ic177q07ja_box_plot.png}
    \caption{Shows How much time spent: Create or edit my own digital content (pictures, videos, music, videos, computer programs), for each cluster, revealed by K-Means, separated by vocational and non-vocational students.}
    \label{fig:time_spent_editing_digital_content}
\end{figure}

\begin{figure}[h!]
    \centering
    \includegraphics[width=0.8\linewidth]{figures/ic183q01ja_box_plot.png}
    \caption{Shows what students Can do with [digital resources]: Search for and find relevant information online, for each cluster, revealed by K-Means, separated by vocational and non-vocational students.}
    \label{fig:find_relevant_information_online}
\end{figure}

\begin{figure}[h!]
    \centering
    \includegraphics[width=0.8\linewidth]{figures/ictres_box_plot.png}
    \caption{Shows if students have ICT Resources (WLE), for each cluster, revealed by K-Means, separated by vocational and non-vocational students.}
    \label{fig:ict_resources}
\end{figure}

\begin{figure}[h!]
    \centering
    \includegraphics[width=0.8\linewidth]{figures/matheff_box_plot.png}
    \caption{Shows students Mathematics self-efficacy (i.e., formal and applied mathematics) (WLE), for each cluster, revealed by K-Means, separated by vocational and non-vocational students.}
    \label{fig:matheff}
\end{figure}

\begin{figure}[h!]
    \centering
    \includegraphics[width=0.8\linewidth]{figures/ocod1_box_plot.png}
    \caption{Shows students' Mother Occupation (ISCO-08 Occupation code), for each cluster, revealed by K-Means, separated by vocational and non-vocational students.}
    \label{fig:mother_occupation_code}
\end{figure}

\begin{figure}[h!]
    \centering
    \includegraphics[width=0.8\linewidth]{figures/oecd_box_plot.png}
    \caption{Shows students' OECD country, for each cluster, revealed by K-Means, separated by vocational and non-vocational students.}
    \label{fig:oecd_country}
\end{figure}

\begin{figure}[h!]
    \centering
    \includegraphics[width=0.8\linewidth]{figures/st005q01ja_box_plot.png}
    \caption{Shows What is the [highest level of schooling] completed by students mother, for each cluster, revealed by K-Means, separated by vocational and non-vocational students.}
    \label{fig:mother_completed_level_of_schooling}
\end{figure}

\begin{figure}[h!]
    \centering
    \includegraphics[width=0.8\linewidth]{figures/st059q02ja_box_plot.png}
    \caption{Shows student's Total number of [class periods] per week for all subjects, including mathematics, for each cluster, revealed by K-Means, separated by vocational and non-vocational students.}
    \label{fig:student_workload_per_subject_per_week}
\end{figure}

\begin{figure}[h!]
    \centering
    \includegraphics[width=0.8\linewidth]{figures/st250q02ja_box_plot.png}
    \caption{Shows Which of the following are in your [home]: A computer (laptop, desktop, or tablet) that you can use for school work, for each cluster, revealed by K-Means, separated by vocational and non-vocational students.}
    \label{fig:student_at_home_resources_for_school_work}
\end{figure}

\begin{figure}[h!]
    \centering
    \includegraphics[width=0.8\linewidth]{figures/st251q06ja_box_plot.png}
    \caption{Shows How many of these items are there at your [home]: Musical instruments (e.g. guitar, piano, [country-specific example]), for each cluster, revealed by K-Means, separated by vocational and non-vocational students.}
    \label{fig:musical_intrument_at_home}
\end{figure}

\begin{figure}[h!]
    \centering
    \includegraphics[width=0.8\linewidth]{figures/st253q01ja_box_plot.png}
    \caption{Shows How many [digital devices] with screens are there in your [home], for each cluster, revealed by K-Means, separated by vocational and non-vocational students.}
    \label{fig:digital_devices_with_screens_at_home}
\end{figure}

\begin{figure}[h!]
    \centering
    \includegraphics[width=0.8\linewidth]{figures/st255q01ja_box_plot.png}
    \caption{Shows How many books are there in your [home], for each cluster, revealed by K-Means, separated by vocational and non-vocational students.}
    \label{fig:how_many_books_at_home}
\end{figure}

\begin{figure}[h!]
    \centering
    \includegraphics[width=0.8\linewidth]{figures/st256q03ja_box_plot.png}
    \caption{Shows How many of these books at [home]: Contemporary literature, for each cluster, revealed by K-Means, separated by vocational and non-vocational students.}
    \label{fig:how_many_contemporary_literature_books_at_home}
\end{figure}

\begin{figure}[h!]
    \centering
    \includegraphics[width=0.8\linewidth]{figures/st289q02ja_box_plot.png}
    \caption{Shows How familiar are you with the following mathematical terms: Area of a circle, for each cluster, revealed by K-Means, separated by vocational and non-vocational students.}
    \label{fig:familiarity_with_area_of_circle}
\end{figure}

\begin{figure}[h!]
    \centering
    \includegraphics[width=0.8\linewidth]{figures/st289q05wa_box_plot.png}
    \caption{Shows How familiar are you with the following mathematical terms: Linear equation, for each cluster, revealed by K-Means, separated by vocational and non-vocational students.}
    \label{fig:familiarity_with_linear_equation}
\end{figure}

\begin{figure}[h!]
    \centering
    \includegraphics[width=0.8\linewidth]{figures/st289q06ja_box_plot.png}
    \caption{Shows How familiar are you with the following mathematical terms: Pythagorean theorem, for each cluster, revealed by K-Means, separated by vocational and non-vocational students.}
    \label{fig:familiarity_with_pythagorean_theorem}
\end{figure}

\begin{figure}[h!]
    \centering
    \includegraphics[width=0.8\linewidth]{figures/st289q10wa_box_plot.png}
    \caption{Shows How familiar are you with the following mathematical terms: Probability, for each cluster, revealed by K-Means, separated by vocational and non-vocational students.}
    \label{fig:familiarity_with_probability}
\end{figure}

\begin{figure}[h!]
    \centering
    \includegraphics[width=0.8\linewidth]{figures/st290q01wa_box_plot.png}
    \caption{Shows How confident in math tasks: Working out from a [train timetable] how long it would take to get from one place to another, for each cluster, revealed by K-Means, separated by vocational and non-vocational students.}
    \label{fig:confidence_with_train_timetable}
\end{figure}

\begin{figure}[h!]
    \centering
    \includegraphics[width=0.8\linewidth]{figures/st290q02wa_box_plot.png}
    \caption{Shows How confident in math tasks: Calculating how much more expensive a computer would be after adding tax, for each cluster, revealed by K-Means, separated by vocational and non-vocational students.}
    \label{fig:confidence_with_calculating_sales_tax}
\end{figure}

\begin{figure}[h!]
    \centering
    \includegraphics[width=0.8\linewidth]{figures/st290q03wa_box_plot.png}
    \caption{Shows How confident in math tasks: Calculating how many square metres of tiles you need to cover a floor, for each cluster, revealed by K-Means, separated by vocational and non-vocational students.}
    \label{fig:confidence_with_calculating_square_meters}
\end{figure}

\begin{figure}[h!]
    \centering
    \includegraphics[width=0.8\linewidth]{figures/st290q05wa_box_plot.png}
    \caption{Shows How confident in math tasks: Solving an equation like 6x[sup]2[/sup]+5=29, for each cluster, revealed by K-Means, separated by vocational and non-vocational students.}
    \label{fig:confidence_with_solving_quadratic_equation}
\end{figure}

\begin{figure}[h!]
    \centering
    \includegraphics[width=0.8\linewidth]{figures/st291q01ja_box_plot.png}
    \caption{Shows How confident in math tasks: Extracting mathematical information from diagrams, graphs, or simulations, for each cluster, revealed by K-Means, separated by vocational and non-vocational students.}
    \label{fig:confidence_with_extracting_mathematical_information_from_visual_assets}
\end{figure}

\begin{figure}[h!]
    \centering
    \includegraphics[width=0.8\linewidth]{figures/st355q02ja_box_plot.png}
    \caption{Shows Confident can do in future: Using a video communication program (e.g. Zoom™, Skype™, Google® Meet™, Microsoft® Teams), for each cluster, revealed by K-Means, separated by vocational and non-vocational students.}
    \label{fig:confidence_with_using_video_conference_tools}
\end{figure}

\begin{figure}[h!]
    \centering
    \includegraphics[width=0.8\linewidth]{figures/feature_importances_cluster_0.png}
    \caption{Shows the Top Feature Importances for Cluster 0, revealed by Random Forest Classifier (vocational vs non-vocational students)}
    \label{fig:feature_importances_cluster_0}
\end{figure}

\begin{figure}[h!]
    \centering
    \includegraphics[width=0.8\linewidth]{figures/feature_importances_cluster_1.png}
    \caption{Shows the Top Feature Importances for Cluster 1, revealed by Random Forest Classifier (vocational vs non-vocational students)}
    \label{fig:feature_importances_cluster_1}
\end{figure}

\begin{figure}[h!]
    \centering
    \includegraphics[width=0.8\linewidth]{figures/feature_importances_cluster_2.png}
    \caption{Shows the Top Feature Importances for Cluster 2, revealed by Random Forest Classifier (vocational vs non-vocational students)}
    \label{fig:feature_importances_cluster_2}
\end{figure}

%% else use the following coding to input the bibitems directly in the
%% TeX file.

%%\begin{thebibliography}{00}

%% \bibitem[Author(year)]{label}
%% For example:

%% \bibitem[Aladro et al.(2015)]{Aladro15} Aladro, R., Martín, S., Riquelme, D., et al. 2015, \aas, 579, A101


%%\end{thebibliography}

\end{document}

\endinput
%%
%% End of file `elsarticle-template-harv.tex'.